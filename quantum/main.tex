\documentclass{article}

\begin{document}
\title{Quantum computer architecture}
\author{John Scott, Oliver Thomas}
\maketitle

\abstract{This is for a demonstration of 12 or 16 \textit{Classical qubits} to
    demonstrate the differences between a 4 bit digital microcomputer and state of the
art quantum processors}

\section{Notes}

\begin{itemize}
\item 16 qubits (4 by 4 grid) requires $2^16=65536$ entries in the state vector. Each entry must be signed and could optionally be complex (to express the addition algorithm). If we used 4 bits of precision that means complex amplitudes require 16bits. Then the state vector can be stored in 131072 Bytes (approx. 131 KiB).
\item Multiplication requires only 2 by 2 and 4 by 4 matrix multiplications. The unitary operations can be performed in block diagonal form. You probably need 2 or 3 external memory chips (one for the state vector and a few for working memory).
\item States of qubits will be shown using RGB LEDs. We also need to figure out how to measure.
\item The memory needs to be quite fast. We found one (AS6C4008-55PCN) that has 55ns read/write times (we think). The time it takes the micro-controller to do the matrix multiplication is also important. If the micro-controller has 70MHz instruction rate and a single matrix multiplication takes 200 clock cycles for a 4 by 4 matrix (guess) then the total processing will take about 40ms assuming that you need to do $2^14$ blocks. That seems quick enough.
  \item We could cycle between the non-zero amplitudes to show the superposition. If you did it fast enough in proportion to different amplitude sizes it would do the averaging for you
\end{itemize}

\end{document}
